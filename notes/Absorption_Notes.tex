\documentclass[12pt]{article}
\input{MyPreamble}
\usepackage{multirow}
\usepackage{amsmath}
\usepackage{fancyhdr}
\usepackage[left=1in,top=1.0in,right=1in,headsep=0.25in,footskip=0.3in,bottom=0.7in]{geometry}
%\usepackage[left=1in,top=1.2in,right=1in,footskip=0.3in,bottom=0.7in,showframe]{geometry}
%\usepackage[left=1in,top=1in,right=1in,bottom=1in,nohead]{geometry}
\usepackage{graphicx}
\usepackage{gensymb}
\renewcommand{\arraystretch}{1} % spacing between table rows
\usepackage[]{caption}
\setlength{\abovecaptionskip}{0pt}
\setlength{\belowcaptionskip}{-5pt}
\setlength{\intextsep}{10pt plus 2pt minus 2pt}
\usepackage[normalem]{ulem}
\newcounter{Labcounter}
\newcounter{Taskcounter}
\numberwithin{equation}{section}
\fancyhf{}
\pagestyle{fancy}
\fancypagestyle{plain}{ %
% \fancyhf{} % remove everything
\renewcommand{\headrulewidth}{0pt} % remove lines as well
%\renewcommand{\footrulewidth}{0pt}
%\cfoot{Page \thepage~of \pageref{LastPage}}}
\cfoot{\thepage}
}
\usepackage[numbered,framed]{matlab-prettifier}
\lstMakeShortInline[style=Matlab-editor]"

\usepackage[inline,shortlabels]{enumitem}
%\usepackage{paralist}
\def\code#1{\texttt{#1}}

\newcommand{\blue}[1]{\textcolor{blue}{#1}} %for displaying red texts
%\newcommand{\rood}[1]{} %for displaying red texts
\newcommand{\red}[1]{\textcolor{red}{[#1]}} %for displaying red texts


\lhead{\textit{Nanospheres...}}
%\chead{\Large \textbf{Paul W.~Leu } \vspace{0.3em}}
%\chead{\Large \textbf{NSF Biographical Sketch: Paul W.~Leu} \vspace{0.3em}}
\rhead{\textit{Leu}}


\newcommand{\vectornorm}[1]{\left|\left|#1\right|\right|}
%\usepackage[top=2.5cm, bottom=2.5cm, left=2.5cm, right=2.5cm]{geometry}
\usepackage[normalem]{ulem}
\newenvironment{packed_enum}{
\begin{enumerate}
  \setlength{\topsep}{0pt}
  \setlength{\partopsep}{0pt}
  \setlength{\itemsep}{1pt}
  \setlength{\parskip}{0pt}
  \setlength{\parsep}{0pt}
}{\end{enumerate}}

%\usepackage[small]{caption}
\usepackage[draft]{pdfcomment}
\usepackage{wrapfig}
\usepackage{hyperref}
%\usepackage{paralist}
\usepackage{amsmath}
\usepackage{amssymb}
\usepackage{amsfonts}
\usepackage{textcomp}
\usepackage{subfig}
\usepackage{framed}
\usepackage{setspace}
\usepackage{here}
\usepackage[numbers, square, comma, sort&compress]{natbib}

\usepackage[compact]{titlesec}
\titlespacing{\section}{0pt}{0ex}{0pt}
\titlespacing{\subsection}{0pt}{0pt}{0pt}
\usepackage{xcolor}

\usepackage[]{caption}
\setlength{\abovecaptionskip}{0pt}
\setlength{\belowcaptionskip}{-5pt}
\setlength{\intextsep}{10pt plus 2pt minus 2pt}


\usepackage{float}
\floatstyle{plaintop}
\newfloat{program}{thp}{lop}
\floatname{program}{Table}

\newfloat{wrapprogram}{thp}{lop}
\floatname{wrapprogram}{Table}

%\setlength{\intextsep}{10pt plus 2pt minus 2pt}
% bold face: highlight keywords, or big ideas
% italics: inconspicuous stressing of key points
% underline: hypothesis; avoid

\usepackage{bm}

\date{\today }

\author{
Paul W. Leu\\
University of Pittsburgh\\
Pittsburgh, PA}






%\def\myTitle{CAREER: Transforming Solar Energy Harvesting through Nanophotonic Light Trapping}
%\def\myTitle{CAREER: Characterizing Enhanced Absorption and Carrier Collection Mechanisms in Silicon and Zinc Oxide Nanocone-based Solar Cells}
%Upper bound
%   100 hours/year
%    40 hours
%    
%\title{Ultimate Limits of Silicon Nanostructures for Photon Management\\
%\title{Determining Silicon/Metal Nanostructures that Approach the Wave-Optics Light Trapping Limit by Data Mining of Electrodynamic Simulations}
%\title{CAREER: Characterizing Enhanced Absorption and Carrier Collection Mechanisms in Silicon and Zinc Oxide Nanocone-based Solar Cells}
%\title{Predicting Wave-Optics Light Trapping in Silicon and Metal Nanostructure by Data Mining of Electrodynamic Simulations}
%Keywords: ultra thin, light trapping, wave-optics light trapping, coherent light trapping, optimization framework, data mining}


\graphicspath{{../}{../Figures/}{Figures/}}

%\title{Modeling and Manufacturing of Three Dimensional Silicon Nanostructures for Photon Management}



\begin{document}

\tableofcontents

\section{Notes}




\begin{align}
    E_{\text{TM}}(z) &= A_T^+ e^{-j k_z z} + A_T^- e^{j k_z z} \notag \\
    H_{\text{TM}}(z) &= \frac{1}{\eta_{\text{TM}}} \left[ A_T^+ e^{-j k_z z} - A_T^- e^{j k_z z} \right] \tag{7.2.9} \\
    E_{\text{TE}}(z) &= B_T^+ e^{-j k_z z} + B_T^- e^{j k_z z} \notag \\
    H_{\text{TE}}(z) &= \frac{1}{\eta_{\text{TE}}} \left[ B_T^+ e^{-j k_z z} - B_T^- e^{j k_z z} \right] \tag{7.2.10}
\end{align}

For TM, we have 
$P_z = \frac{\operatorname{Re}[E_x H_y^*]}{2}$
and for TE, 
$P_z = -\frac{\operatorname{Re}[E_y H_x^*]}{2}.$

\begin{equation}
    P_z = \frac{1}{2\eta_T} \left( \lvert E_T^+ \rvert^2 - \lvert E_T^- \rvert^2 \right) \tag{7.3.17}
\end{equation}

%\begin{enumerate}
%\item Get the Poynting vector and then integrate over the thickness of the layer.  
%\item Normalize over incoming power.
%\end{enumerate}

For oblique incidence, the absorption is
\begin{equation}
A = \frac{\omega}{2} \epsilon_0 n'' \left[ 
\frac{|E_{film}^f|^2 (1 - e^{-2k''d})}{2k''} 
+ \frac{|E_{film}^b|^2 (e^{2k''d} - 1)}{2k''} 
+ \frac{\text{Re}(E_{film}^f (E_{film}^b)^*) \sin(2k'd)}{k'} 
\right]
\end{equation}

For TM, I only have the transverse electric field.  Thus, I need to use 
\begin{equation}
A = \frac{\omega}{2 \cos^2 \theta} \epsilon_0 n'' \left[ 
\frac{|E_{film}^f|^2 (1 - e^{-2k''d})}{2k''} 
+ \frac{|E_{film}^b|^2 (e^{2k''d} - 1)}{2k''} 
+ \frac{\text{Re}(E_{film}^f (E_{film}^b)^*) \sin(2k'd)}{k'} 
\right]
\end{equation}

To normalize the absorption, 
\begin{equation}
A = k_0 n'' \left[ 
\frac{|E_{film}^f|^2 (1 - e^{-2k''d})}{2k''} 
+ \frac{|E_{film}^b|^2 (e^{2k''d} - 1)}{2k''} 
+ \frac{\text{Re}(E_{film}^f (E_{film}^b)^*) \sin(2k'd)}{k'} 
\right]
\end{equation}

 
 
 Within a given layer, absorption is an analytical function:
\begin{equation}
a(z) = A_1 e^{2z \, \text{Im}(k_z)} + A_2 e^{-2z \, \text{Im}(k_z)} + A_3 e^{2iz \, \text{Re}(k_z)} + A_3^* e^{-2iz \, \text{Re}(k_z)}
\end{equation}

 
\begin{align}
\int_0^d a(z) \, dz = \frac{A_1}{2 \, \text{Im}(k_z)} \left( e^{2d \, \text{Im}(k_z)} - 1 \right)
+ \frac{A_2}{-2 \, \text{Im}(k_z)} \left( e^{-2d \, \text{Im}(k_z)} - 1 \right) \\
+ \frac{A_3}{2i \, \text{Re}(k_z)} \left( e^{2id \, \text{Re}(k_z)} - 1 \right)
+ \frac{A_3^*}{-2i \, \text{Re}(k_z)} \left( e^{-2id \, \text{Re}(k_z)} - 1 \right).
\end{align}

 
 






\bibliographystyle{IEEEtran}
\bibliography{AllRefs,LAMPPapers}


\end{document}
I